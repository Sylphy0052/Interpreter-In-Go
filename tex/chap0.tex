\section{はじめに}

インタプリタ: 他のプログラムを入力として受け取って何かしらを出力するプログラム

Brainfuck: 8つの命令だけから構成されるプログラミング言語

精巧なインタプリタ: 最適化されていて,構文解析と評価において高度な技術を採用している

JIT(Just In Time)インタプリタ: 入力を都度ネイティブの機械語に翻訳し実行する

tree-walking型インタプリタ: ソースコードを構文解析し,抽象構文木(Abstract Syntax Tree: AST)を構築し,この木を評価する

\begin{itemize}
  \item 字句解析器
  \item 構文解析器
  \item 木表現
  \item 評価器
\end{itemize}

\subsection{Monkeyプログラミング言語とインタプリタ}

Monkey言語の特徴

\begin{itemize}
  \item C言語風の構文
  \item 変数束縛
  \item 整数と真偽値
  \item 算術式
  \item 組み込み関数
  \item 第一級の高階関数
  \item クロージャ
  \item 文字列データ型
  \item 配列データ型
  \item ハッシュデータ型
\end{itemize}

\begin{lstlisting}[caption=Monkeyで名前を値に束縛]
let age = 1;
let name = "monkey";
let result = 10 * (20 / 2);
\end{lstlisting}

\begin{lstlisting}[caption=整数の配列とハッシュに名前を束縛]
let myArray = [1, 2, 3, 4, 5];
let thorsten = {"name": "Thorsten", "age": 28}

myArray[0] // => 1
thorsten["name"] // => "Thorsten"
\end{lstlisting}

\begin{lstlisting}[caption=関数に名前を束縛]
let add = fn(a, b) { return a + b; };
let add = fn(a, b) { a + b; }; // returnを省略
add(1, 2);
\end{lstlisting}

\begin{lstlisting}[caption=フィボナッチ数]
let fibonacci = fn(x) {
  if (x == 0) {
    0
  } else {
    if (x == 1) {
      1
    } else {
      fibonacci(x - 1) + fibonacci(x - 2);
    }
  }
};
\end{lstlisting}

\begin{lstlisting}[caption=高階関数, label=first-class_function]
let twice = fn(f, x) {
  return f(f(x));
}

let addTwo = fn(x) {
  return x + 2;
}

twice(addTwo, 2); // => 6
\end{lstlisting}

第一級関数: 関数を関数呼び出しの引数として使うことができる(ソースコード\ref{first-class_function})

REPL(Read(読み込み)/Evaluate(評価)/Print(表示)/Loop(繰り返し): 対話的実行環境)でMonkeyのソースコードをトークナイズし,構文解析し,抽象構文木と呼ばれるコードの内部表現を生成し,それからその木を評価する.

インタプリタに必要な機能

\begin{itemize}
  \item 字句解析器
  \item 構文解析器
  \item 抽象構文木(AST)
  \item 内部オブジェクトシステム
  \item 評価器
\end{itemize}

\subsection{どうしてGoか?}

Go言語は読むのが簡単で理解しやすい.

ツール
\begin{itemize}
  \item gofmt: Goの統一されたコーディングスタイル
  \item 組み込みのテスティングフレームワーク
  \item 他の言語でもほとんど同じように書ける
\end{itemize}
